\section{Implementation}
\label{sec:implementation}

For implementation, \libipc is divided to two parts: monitor and userspace library. Both part are implemented in C/CPP with total 5000 lines of code. We take advantage of templates of CPP for different types of queues in our design. Specifically, we have two techniques to highlight:
\subsection{Seamless system call hook}
\label{subsec:syscall-hook}
In order to support existing applications seamlessly, \libipc leverages \textit{LD\_PRELOAD} environment variable in Linux to load a shared library to applications, which replace the function of GNU LibC which are the wrappers of the system call. 

In our system, we use the range of file descriptor number to identify whether the system call should be hook to \libipc to avoid problems related to file descriptors not related to socket. Taken the idea of \cite{huang2017high}, in \libipc, FD number is allocated in a top-down way i.e. from $2^{31}-1$ while Linux assign the number from bottom to up i.e. from zero. We set $2^{30} - 1 $ as the threshold to distinguish whether the system call ought to be handled by \libipc.


% \textbf{
% Separate file descriptors in \sys and Linux. Linux use low FD space, \sys use % high FD space (what FD space). Same as LOS~\cite{huang2017high}.
% }



\subsection{High performance lockless ring-buffer}


% Inter-process communication through single-writer and single-reader lockless % shared-memory queue. The queue is a ring buffer.
% 
% Multiplex file descriptors through a same queue to reduce memory footprint and % polling cost. Why we can do this without head-of-line blocking problem? Because % most applications call epoll to get events and process them one by one.
% 

Access local storage. Use SPDK and user-mode file system (cite). How to multiplex processes in accessing a file system? (1) directory and metadata go to monitor, (2) read/write within allocated area of a file: process self, (3) append or read/write outside allocated area: handled by master process of a shared file. Monitor pre-allocate free blocks to processes (batch allocation and free).

Discuss limitations: order (linux is sequentially consistent, ours is FIFO).