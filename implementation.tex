\section{Implementation}
\label{sec:implementation}

For implementation, \libipc is divided to two parts: monitor and userspace library. Both parts are implemented in $\approx$5000 lines of C/C++ code. We take advantage of C++ templates for different types of queues in our design. %Specifically, we have two techniques to highlight:
In this section, we highlight several techniques in implementation:

%\subsection{Seamless system call hook}
%\label{subsec:syscall-hook}

\textbf{LD\_PRELOAD to intercept Linux APIs.}
In order to support existing applications seamlessly, \libipc leverages \textit{LD\_PRELOAD} environment variable in Linux to load a shared library to applications, which replace the function of GNU LibC which are the wrappers of the system call. 

\textbf{Multiplex FD between kernel and \libipc{}.}
In our system, we use the range of file descriptor number to identify whether the system call should be hook to \libipc to avoid problems related to file descriptors not related to socket. Taken the idea of \cite{huang2017high}, in \libipc, FD number is allocated in a top-down way i.e. from $2^{31}-1$ while Linux assign the number from bottom to up i.e. from zero. We set $2^{30} - 1 $ as the threshold to distinguish whether the system call ought to be handled by \libipc.


% \textbf{
% Separate file descriptors in \sys and Linux. Linux use low FD space, \sys use % high FD space (what FD space). Same as LOS~\cite{huang2017high}.
% }

\textbf{Accelerate access to local storage.}
Use SPDK and user-mode file system (cite). How to multiplex processes in accessing a file system? (1) directory and metadata go to monitor, (2) read/write within allocated area of a file: process self, (3) append or read/write outside allocated area: handled by master process of a shared file. Monitor pre-allocate free blocks to processes (batch allocation and free).

\textbf{Limitations.}
Ordering of multiple senders (linux is sequentially consistent). Applications that do not use GNU libc. Linux AIO and Windows I/O completion ports. /proc fs.