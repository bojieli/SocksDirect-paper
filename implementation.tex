\section{Implementation}
\label{sec:implementation}

How many lines of code...

The implementation section discusses some less important details in the design.


For compatibility
LD\_PRELOAD.
Separate file descriptors in \sys and Linux. Linux use low FD space, \sys use high FD space. Same as LOS (APNet'17).

\subsection{Lockless Shared-memory Queue}
\label{subsec:lockless-queue}

Inter-process communication through single-writer and single-reader lockless shared-memory queue. The queue is a ring buffer.

Multiplex file descriptors through a same queue to reduce memory footprint and polling cost. Why we can do this without head-of-line blocking problem? Because most applications call epoll to get events and process them one by one.

Data buffer and free slot allocation.

One-sided  RDMA: credit-based ring buffer. Intra-server: valid-bit based ring buffer.


Thread can access file descriptors created by other threads. Resolution: give each thread a unique file descriptor space (similar to thread stack). If a thread access a file descriptor owned by another thread, process it like pthread\_create.

Access local storage. Use SPDK and user-mode file system (cite). How to multiplex processes in accessing a file system? (1) directory and metadata go to monitor, (2) read/write within allocated area of a file: process self, (3) append or read/write outside allocated area: handled by master process of a shared file. Monitor pre-allocate free blocks to processes (batch allocation and free).

Discuss limitations: order (linux is sequentially consistent, ours is FIFO).