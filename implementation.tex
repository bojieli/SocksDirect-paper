\iffalse
\section{Implementation}
\label{sec:implementation}

For implementation, \libipc is divided to two parts: monitor and userspace library. Both parts are implemented in $\approx$5000 lines of C/C++ code. We take advantage of C++ templates for different types of queues in our design. %Specifically, we have two techniques to highlight:
In this section, we highlight several techniques in implementation:

%\subsection{Seamless system call hook}
%\label{subsec:syscall-hook}

\parab{LD\_PRELOAD to intercept Linux APIs.}
\libipc uses \textit{LD\_PRELOAD} environment variable in Linux to load a shared library and intercept the system call wrappers of GNU libc.

\parab{Multiplex FD between kernel and \libipc{}.}
Taking the idea of MegaPipe~\cite{han2012megapipe} and LOS~\cite{huang2017high}, we partition the FD space between \libipc and Linux kernel. Linux assigns FD from zero to $2^{30}-1$, while \libipc assigns FD from $2^{31}-1$ down to $2^{30}$.

\parab{Multiplex events between kernel and \libipc{}.}
The FD set of \texttt{epoll} may include both sockets and other \textit{kernel FDs} handled by Linux kernel.
%LOS~\cite{huang2017high} periodically invokes non-blocking \texttt{epoll\_wait} syscall with kernel FDs, which leads to a trade-off between delay and CPU overhead. Differently,
\libipc{} creates a per-process \textit{epoll thread} which runs an infinite loop of blocking \texttt{epoll\_wait} syscall with kernel FDs. Whenever epoll thread receives a kernel event, it broadcasts the event to application threads via shared memory queues. \texttt{Epoll\_wait} in \libipc{} will return such kernel events in addition to socket events. Note that Linux allows an event to be received by multiple threads sharing the FD.

%\textbf{Accelerate access to local storage.}
%Use SPDK and user-mode file system (cite). How to multiplex processes in accessing a file system? (1) directory and metadata go to monitor, (2) read/write within allocated area of a file: process self, (3) append or read/write outside allocated area: handled by master process of a shared file. Monitor pre-allocate free blocks to processes (batch allocation and free).
\fi
