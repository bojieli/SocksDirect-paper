\section{Architecture}
\label{sec:design}

This subsection shows the architecture of \sys. (Library Operating System for Socket)

Processes as a distributed system.
Threads are treated as separate processes.
Processes communicate with each other via message passing.
Shared-memory isolation.

File system, inter-process communication, network stack and storage stack in user-mode.
Use unmodified Linux kernel for process/thread creation, scheduling and memory management.

To remove coordination bottleneck:
Separate scalable and non-scalable parts of socket API.
Scalable parts are implemented in the library in each process, and non-scalable parts are delegated to a monitor process.
Monitor is like the coordinator in distributed system.
For most socket operations, two processes communicate with each other directly, without involving the monitor process.



\section{Inter-Process Socket in a Server}
\label{sec:intra-server}

\subsection{Scaling Socket API}

Data structure in each process. Data structure in the monitor process.

Need a figure to show the lifecycle (message passing flow) of a socket connection. Three sub-figures:

\begin{itemize}
	\item Socket initialization.
	\item Connection setup.
	\item Data transmission.
\end{itemize}


\subsection{Fork and Thread Creation}

Key point: Coordination in shared socket.

\subsection{Lockless Shared-memory Queue}

Inter-process communication through single-writer and single-reader lockless shared-memory queue. The queue is a ring buffer.

Multiplex file descriptors through a same queue to reduce memory footprint and polling cost. Why we can do this without head-of-line blocking problem? Because most applications call epoll to get events and process them one by one.


Head of line blocking problem implies additional features:
\begin{itemize}
	\item Fetch from middle (del).
	\item One normal queue and an emergency queue. Why we need the emergency queue.
	\item Overflow per-file-descriptor slot to prevent deadlock.
\end{itemize}

Data buffer and free slot allocation.

\subsection{Event Polling and Notification}

Discuss epoll and signal.

Message passing may wait too long.
Signal mechanism. Signal process after waiting until timeout.

Monitor detect process death (via control socket SIGHUP) and notify endpoints.


\subsection{Zero Copy}



\section{Utilizing RDMA for Inter-server Socket}
\label{sec:rdma}

