\section{Introduction}
\label{sec:intro}

%Most cloud applications use the socket API for inter-process communication among components or containers inside a same server and across data center network, in addition to serving Internet users. For example, communication intensive applications (\textit{e.g.} nginx and memcached) spend 50\%$\sim$90\% of CPU time in the OS kernel, mostly processing socket operations.
%The overhead of Linux socket attributes to kernel crossing in system calls, context switch, process scheduling, synchronization, memory copy, cache miss and TCP transport.
%Applications in high performance computing have a long tradition of using shared memory for intra-server communication and RDMA for inter-server, thus avoiding the overheads above.
%However, these abstractions are radically different from socket, so it is complicated and potentially insecure to port socket applications to shared memory and RDMA.
%The overhead of Linux socket becomes salient given the rapid growth of network speed and number of CPU cores per server. %We benchmark communication intensive applications (\textit{e.g.} Nginx and memcached) and find that 50\%$\sim$90\% of CPU time is spent in the OS kernel. When more CPU cores are utilized, they even spend a larger portion of time in the kernel~\cite{boyd2010analysis}. In addition to CPU overhead, latency is also a problem. The round-trip time (RTT) between two processes communicating with shared memory can be as low as 0.2$\mu$s, while TCP socket RTT between two cores is $\approx$16$\mu$s. In a data center, the RTT between RDMA servers is also one order of magnitude lower than kernel TCP/IP stack.

Socket API is the most widely used communication primitive in modern OS. It is used universally for communications between processes, containers and hosts.
Traditional Linux socket implementation is not optimal. It can only achieve latency and throughput numbers an order of magnitude worse than what the raw hardware is capable of.
%Socket operations have latency and throughput one order of magnitude worse than the raw hardware.
%Socket operations are expensive and scale poorly with modern multi-core CPUs.
Communication intensive applications such as distributed key-value stores and web servers could spend 50\%$\sim$90\% of CPU time in the OS kernel, mostly processing socket operations.

There has been extensive work aiming at reducing the socket overhead, but existing approaches are not satisfactory.
First, many user-space sockets are not fully compatible with native Socket in areas such as when a process forks, when multiple processes listen to the same port, and when communication is intra-host.
Second, some user-space sockets have security issues because applications can directly access shared NIC queues, thus violate process isolation and access control policies.
Those solutions that do preserve security often incur the overhead of kernel crossing or virtual switch.
Third, even in the performance front, there are still much room for improvement. For example, none of existing works can achieve performance close to raw RDMA, 
because they fail to remove some important overheads such as multi-thread synchronization, memory copy and TCP/IP packet processing.
%In addition, when the number of active threads is larger than that of CPU cores, polling and kernel event notification are a dilemma.
%Finally, they cannot take advantage of low latency networking provided by RDMA.
%These approaches still have limitations, either not fully removing some of the important overheads such as context switch and synchronization~\cite{lin2016scalable,han2012megapipe,jeong2014mtcp,baumann2009multikernel}, not fully compatible with Linux socket API, or cannot take advantage of modern networking hardware capabilities such as RDMA~\cite{dunkels2001design,jeong2014mtcp,libvma,openonload}.
%There has been extensive work aiming to release the bare metal performance of multi-core CPU and data center network. For intra-server communication, there are mainly three lines of research. The first category of work use the NIC as a switch~\cite{peter2016arrakis,belay2017ix,yasukata2016stackmap}, but going deep to the NIC introduces $\approx2 \mu$s delay due to PCIe latency, one order of magnitude higher than shared memory. A second line of work optimize or redesign the kernel socket stack~\cite{lin2016scalable,han2012megapipe,jeong2014mtcp,baumann2009multikernel}, where the kernel uses peer-to-peer shared memory communication among cores. However, this approach does not eliminate context switch overhead, while system call batching introduces extra latency. Some other works use dedicated cores as a virtual switch~\cite{huang2017high}, which limits multi-core scalability.

%For inter-server communication, most works leverage a user-space stack~\cite{dunkels2001design,jeong2014mtcp,libvma,openonload} to achieve kernel bypass, but the CPU still needs to handle reliable transport. As RDMA becomes widely available in data centers, we hope to offload the transport to RDMA NICs when the peer supports RDMA. Furthermore, most works assume only one connection per pair of processes. However, load balancers, web servers and application gateways serve many concurrent connections~\cite{nishtala2013scaling,lin2016scalable,belay2017ix}. In light of this, both connection setup, event notification and data transmission under high concurrency need to be efficient.

%To demonstrate high performance, most existing works propose new abstractions for inter-process and inter-server communication.  existing socket applications need modifications to use the new abstractions. Furthermore, these stacks are not optimized for a large number of concurrent or short-lived connections, which is an important workload to serve Internet users and large distributed systems.

%One line of research optimize the kernel code or design user-space compatible stacks for higher socket performance. The kernel optimization approach does not eliminate context switch overhead, while system call batching introduces extra latency. User-space stacks are mostly designed for inter-server connections. With the trend of containerized micro-services, we expect an increasing number of applications or containers to be hosted on each server, where inter-process communication (IPC) inside server has more significance.

%To simplify deployment, we hope to accelerate existing applications without modification to the code. \textit{Socket compatibility} adds another dimension of challenge. The socket interface was designed for networking and IPC in millisecond scale, when memory copy, context switch, synchronization, cache miss and cache migration were considered inexpensive~\cite{barroso2017attack,belay2017ix}. An efficient socket architecture for microsecond-scale networking and IPC requires minimizing all overheads above. The semantics lead to challenges. First, the send buffer can be modified by application after non-blocking \texttt{send}, and the receive buffer is not determined until application calls \texttt{recv}. Data copy on \texttt{send} and \texttt{recv} seems mandatory. Second, connections are shared by processes and threads after \texttt{fork} and thread creation. It is challenging to avoid synchronization in this multi-producer and multi-consumer FIFO model. Third, multiple processes listening on a same IP and port compete for incoming connections.

%\textbf{The above part is motivation and related work.}

We design \sys{}, a user-space socket architecture with compatibility, security and performance in mind.
\begin{ecompact}
	\item \textbf{Compatibility}.
	Existing applications can use \sys{} as a drop-in replacement with no modification.
	It supports both intra-host and inter-host communication, and behaves correctly during process fork and thread creation.
	If remote peer does not support \sys{}, the system falls back to TCP transparently.
	\item \textbf{Security}.
	\sys{} preserves isolation among applications and containers, and it enforces firewall rules and access control policies.
	\item \textbf{High Performance}.
	\sys{} delivers consistent high throughput and low latency that is scalable with number of CPU cores, and the performance does not degrade significantly with vast number of concurrent connections.
\end{ecompact}



%We design \sys{}, a high performance user-space socket architecture that is compatible with existing applications and preserves isolation among processes, while being scalable to multiple cores, threads and concurrent connections.
%\sys is designed to be compatible with existing applications using Linux socket.
%It preserves Linux socket semantics, and in particular behaves correctly with process fork and thread creation. %We also take particular caution to preserve \textit{process isolation} in \sys.
%The main component of \sys{} is a user-space library \libipc{} that intercepts Linux \emph{glibc} APIs, implements socket operations in user space and forwards others to the kernel.
%Applications can take advantage of \libipc{} by simply using the \emph{LD\_PRELOAD} environment variable in Linux to load the library.

%The emergence of containers and micro-services draws attention to intra-server communication~\cite{bailis2016introducing}.
%%For manageability and fault isolation, developers break monolithic services into self-contained microservice containers, interconnected via container network. 
%Following the trend of recent data-plane operating systems~\cite{peter2016arrakis,belay2017ix,freeflow}, 
%\sys should optimize for both intra-server and inter-server socket.

%We set out to build \sys{} with five performance goals.
%The first two are \textit{low latency} and \textit{high throughput} (aka low CPU overhead) for both intra- and inter-host connections.
%Moreover, the performance should be \textit{scalable} on modern multi-core CPUs.
%In addition, the performance should not degrade with large number of \textit{concurrent connections} between two processes.
%Finally, context switch should be efficient when multiple active threads run on a CPU core.
% Many user connections are forwarded from the load balancer to an application~\cite{lin2016scalable}. Some applications create a TCP socket to key-value store for each request they process~\cite{nishtala2013scaling}.



%The key to multi-core scalability and inter-process isolation is to avoid synchronization and state sharing. 
%We further minimize message passing by partitioning socket states to each process. We implement the socket APIs \textit{as locally as possible}. In particular, we allocate file descriptors, buffers and manage socket options in the caller process. For non-local operations, we make the best effort to \textit{minimize centralized coordination and blocking wait}. Data transmission and event multiplexing operations (\textit{e.g.} \texttt{send}, \texttt{recv} and \texttt{epoll}) only need non-blocking access to peer-to-peer communication channels. For connection establishment, each server has a coordinator process, called \textit{monitor}, to load balance incoming connections, setup inter-process queues and enforce ACL policies. Delegation to the monitor is more secure and efficient than inter-process synchronization~\cite{roghanchi2017ffwd}.



At its heart, \sys{} follows the principle of separating control and data plane to achieve both security and performance. We treat processes as a shared-nothing distributed system that communicates via peer-to-peer message queues.
We design a per-host \emph{monitor} daemon as the control plane to enforce access control policies, dispatch new connections, perform address translation for overlay networks and establish transport channel between communication peers.
The data plane is handled by a dynamically loaded user-space library \libipc{}, which implements data transmission and event polling in a peer-to-peer fashion between processes while delegates connection creation to the local monitor.
%Applications can take advantage of \libipc{} by simply using the \emph{LD\_PRELOAD} environment variable in Linux to load the library.
\libipc{} intercepts Linux \emph{glibc}, implements all socket-related functions in user space and forwards other APIs to the kernel.

%\iffalse

We face several challenges designing \sys{} to achieve high performance while maintaining compatibility. 
One challenge is to share a socket among threads and processes without locking. A socket connection is a FIFO channel. 
The most straight-forward approach is to implement each connection as a single queue. However, a socket may be shared by multiple 
senders and receivers, locking is thus needed to protect the shared queue, which significantly slows down the system. To avoid locking overhead, we treat each thread as a separate process and create peer-to-peer queues between each pair of 
communicating threads. In our design, a single socket connection may correspond to multiple queues, and we take special care to preserve
FIFO semantics during fork and thread creation. Another challenge is to maintain performance with vast number of concurrent connections. 
To handle many concurrent connections efficiently, we need to reduce memory footprint and improve memory access locality. 
Rather than maintaining a separate queue for each connection, \sys multiplexes socket connections through a single queue for each pair of communicating threads.
We design the queue carefully to enable fetching from the middle of a queue and solve the head-of-line blocking problem. 

We exploit many techniques to effectively utilize hardware and improve system efficiency. 
For communication within a same host, we design a high performance user space shared memory queue.
For communication among hosts in an RDMA enabled data center, we use RDMA NIC hardware for transport. 
To remove memory copy cost for large messages, we take advantage of \emph{page remapping} to achieve transparent zero copy.
To share a CPU core efficiently among multiple active threads without thread wakeup overhead, we leverage \emph{cooperative multitasking}. 
%\fi

\iffalse

We face many challenges designing \sys{}. 
(1) How to share a socket among threads and forked processes without locking?
(2) How to scale to many concurrent connections?
(3) How to utilize shared memory and RDMA efficiently for intra- and inter-host communication?

In both multi-thread and multi-process scenarios, a connection may be shared by multiple senders and receivers.
Existing approaches need locking to protect shared queue and metadata.
To avoid locking overhead, we treat each thread as a separate process.
%, even if the threads have shared memory address space.
\libipc{} uses thread-specific storage and creates peer-to-peer queues between each pair of communicating threads.
To preserve FIFO semantics, we optimize for the common case while prepare for the worst case, and take special care on fork and thread creation.

To handle many concurrent connections efficiently, we need to save memory footprint and improve spatial locality.
For each pair of threads, \sys multiplexes socket connections through one message queue.
Rather than maintaining a separate buffer for each connection and an event notification queue, we receive events and data from the message queue directly.
Observing the event-driven behavior of applications, in normal case the data in queue is fetched by the application in send order.
We design carefully to enable fetching from the middle of queue and solve the head-of-line blocking problem.

We leverage different transports to push performance to the limits of underlying hardware.
For inter-process and inter-container sockets within a same host, we use shared memory in user space.
For sockets among hosts in an RDMA enabled data center, \sys can transparently determine whether the remote endpoint supports \sys.
we fall back to kernel TCP socket.
We design different queue structures for shared memory and RDMA.
We use batched one-sided RDMA write and amortize polling overhead with shared CQ.
%In \sys, sending a small message involves only one cache migration or one-sided RDMA write.
To remove memory copy for large messages, we use \emph{page remapping} to achieve transparent zero copy.
To share a CPU core efficiently among multiple active threads, \sys uses \emph{cooperative multitasking} to remove thread wakeup overhead.

\fi

%The POSIX socket API was designed for networking and IPC in millisecond scale, leading to two performance challenges. First, connections are shared by processes and threads after \texttt{fork} and thread creation. Linux protects this multi-producer multi-consumer FIFO with locks. To scale a shared socket, we \textit{optimize for the common case and prepare for the worst case}. Senders transmit data via different queues in parallel. To ensure receiver ordering, based on the observation that applications seldom receive concurrently from a shared socket, the sender designates a receiver with exclusive access. We further develop mechanisms to avoid deadlock and starvation, in addition to handling unconsumed buffers during \texttt{fork} and thread creation.

%Second, the send buffer can be modified by application after \texttt{send}, and the receive buffer is not determined until application calls \texttt{recv}. Data copy on \texttt{send} and \texttt{recv} seems mandatory. To avoid memory copy of large buffers, we extend the \textit{page remapping} approach~\cite{thadani1995efficient,chu1996zero}, which enables copy-on-write upon \texttt{send} and remaps send buffer to receiver's virtual address upon \texttt{recv}.
%First, we intercept \texttt{memcpy} of full pages to reduce copy-on-write. Second, we move kernel-based page allocation to user-space while preserving security.
%As a result, we achieve zero copy for both shared memory, RDMA and TCP transport.

\sys{} achieves latency and throughput close to the raw performance achievable from the underlying shared memory queue and RDMA.
On the latency side, \sys{} achieves $0.3\mu$s RTT for intra-host socket, 35x lower than Linux and only $0.05\mu$s higher than a bare-metal shared memory queue. For inter-host socket, \sys{} achieves $1.7\mu$s RTT between RDMA hosts, almost the same as raw RDMA verbs and 17x lower than Linux.
On the throughput side, a single thread can send 23~M intra-host messages per second (20x Linux) or 8~M inter-host (7x Linux). For large messages, with zero copy, a single connection saturates NIC bandwidth. Each thread can establish 1.4~M new connections per second (20x Linux). The performance above is scalable with number of cores, and do not degrade significantly with millions of concurrent connections.

%\textbf{End-to-end performance.}

%We evaluate end-to-end performance of \sys{} using two categories of applications: \textit{network functions} and \textit{web services}. For a multi-core pipelined network function (NF) chain, a socket application achieves comparable performance with a state-of-the-art NF framework~\cite{panda2016netbricks}. We also evaluate \sys{} on a standard web application composed of a load balancer, a web service and a key-value store.
%For an HTTP request that involves multi-round-trip key-value store accesses, \sys{} reduces end-to-end latency by 2/3.

\iffalse
This paper makes the following contributions:
\begin{ecompact}
	\item A Linux compatible, secure and high performance user-space socket system that supports both inter-process, inter-container and inter-host communication.
	\item A per-host monitor daemon for trusted control plane and peer-to-peer queues for scalable data plane.
	\item A multi-sender and multi-receiver lockless queue to fully support fork and multi-thread socket sharing.
	\item A memory efficient message queue that multiplexes multiple sockets and allows fetching from any socket, while using shared memory and RDMA transports efficiently.
\end{ecompact}
\fi