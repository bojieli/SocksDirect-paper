\section*{Abstract}
Communication intensive applications in data center hosts with multi-core CPU and high-speed networking hardware often put considerable stress on the socket system.
%Communication intensive applications on modern computers with multi-core CPU and high-speed networking hardware 
%often put considerable stress on traditional socket implementation. 
Linux socket is implemented in the kernel space with shared data structures that needs concurrency protection, which incurs significant overhead.
%This design incurs significant kernel crossing and locking overhead.
Recent user-space sockets often do not support intra-host communication among containers and applications, or have limitations on compatibility, isolation and scalability with multiple threads and concurrent connections.

In this paper, we describe \sys{}, a high performance socket system that is fully compatible with Linux and can be used as a drop-in replacement with no modification to applications.
\sys{} is implemented in user space to avoid kernel crossing cost and simplify deployment.
Each host runs a monitor daemon to securely process the control plane, while the data plane bypasses the monitor.
We achieve multi-thread scalability by considering threads as a shared-nothing message passing distributed system.
To improve memory locality, we multiplex all connections and event notifications between a pair of threads via one queue.
We use a high performance shared memory queue for intra-host communication.
For inter-host communication, we take advantage of modern RDMA hardware, but can also transparently communicate with regular TCP/IP endpoints.
Together with carefully designed zero-copy mechanism and cooperative multitasking, it removes many overheads of existing socket systems.
Experiment shows that \sys{} achieves 4 to 24x message throughput, 10 to 60x better latency, and over 40x connection setup throughput compared with Linux socket.
