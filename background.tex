\section{Background}
\label{sec:background}

\subsection{Why Socket is Important}

Socket is widely used in common datacenter application etc. Related work Balabala (Imply compatibility is important )

Figure xx shows the percentage of sockets of applications. Some detail explanation for the access pattern for applications

Benchmark applications (intra-server and inter-server), show the portion of socket operation in CPU utilization.

How the socket works. A table show the process of using socket:

\begin{itemize}
	\item Process and thread creation. (fork, pthread\_create)
	\item Socket initialization.
	\item Connection setup.
	\item Data transmission.
\end{itemize}


\subsection{Why Socket is Slow}
Source of Linux socket performance bottleneck:
\begin{itemize}
	\item Context switch
	\item Data copy
	\item OS network stack (overhead for intra-server \& does not leverage RDMA hardware for inter-server)
	\item Lock in kernel (CPU overhead \& not scalable)
\end{itemize}

The Multikernel design: replaces locks in kernel with message passing, but does not remove context switch.

\subsection{Challenges for High Performance Socket}
\label{subsec:challenges}

List related work on fast sockets. They are not compatible with existing applications.

Challenges for high performance sockets:
\begin{enumerate}
	\item Compatible with existing software. Multiplex system calls and file descriptors between sockets and OS files.
	\item Coordination free. A socket can be read by multiple receivers and written by multiple senders. The read ordering.
	\item Kernel bypass. Isolation.
	\item Zero copy. Read/write after send, receive to user-specified address.
	\item Bridging proactive and reactive APIs from socket and RDMA.
\end{enumerate}
